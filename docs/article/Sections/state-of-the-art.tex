\section{Background and Related Work}

In this section, we first examine the different design choices adopted by web 
servers in their internal architectures, and how these choices impact the 
behavior of the web template engines.
Then, in Subsection~2.2, we present the main properties that characterize each 
web template technology approach, along with the advantages and drawbacks that 
result from these characteristics.

%%%%%%%%%%%%%%%%%%%%%%%%%%%%%%%%%%%%%%%%%%%%%%%%%%%%%%%%%%%%%%%%%%%%
%-------------------------------------------------------------------
%%%%%%%%%%%%%%%%%%%%%%%%%%%%%%%%%%%%%%%%%%%%%%%%%%%%%%%%%%%%%%%%%%%%

\subsection{Web Framework Architectures and Approaches to PSSR}

In traditional thread-per-request architectures, each incoming request is
handled by a dedicated thread. The web server maintains a thread pool from
which a thread is assigned to handle each request. As the load increases, the
number of active threads can grow rapidly, potentially exhausting the pool.
This can lead to performance and scalability issues, as the system may become
bogged down by context switching and thread management
overhead~\cite{kant2000scalable}. Spring MVC is one of the most widely used
frameworks that follows this model.

On the other hand, in modern \textit{low-thread} architectures, the server uses
a small number of threads to handle a large number of requests.
\textit{Low-thread} servers, also known as
\textit{event-driven}~\cite{event-driven-servers}, offer a significant
advantage in efficiently managing a high number of concurrent I/O operations
with minimal resource usage.

The \textit{non-blocking} I/O model employed in low-thread servers is
well-suited for handling large volumes of data asynchronously~\cite{Meijer12}.
This combination of low-thread servers and asynchronous data models has
facilitated the development of highly scalable, responsive, and resilient web
applications capable of managing substantial data loads~\cite{Jin15}.
The prominence of this concept increased with the advent of Node.js in
2009, and subsequently, various technologies adopted this approach
in the Java ecosystem, including Netty, Akka HTTP, Vert.X, and Spring WebFlux.
% Among these, Spring WebFlux stands out as the most widely used web framework in
% Java, according to surveys like JetBrains' State of Java report (2021) and
% community metrics from platforms like Github and StackOverflow.
The \textit{non-blocking} I/O model in low-thread servers functions optimally
only when HTTP handlers avoid blocking.
Therefore, HTML templates need to be proficient in dealing with the asynchronous
APIs provided by data models.
While most legacy web templates struggle with asynchronous models, DSLs for HTML
face no such limitations, leveraging all constructions available in the host
programming language.
However, the unexpected intertwining of asynchronous handlers' completion and
HTML builders' execution may potentially lead to malformed HTML with an
unexpected layout, as demonstrated in the subsequent subsection~\ref{sec:async-support}.

An alternative approach involves utilizing user-level threads, while maintaining
a blocking I/O and a synchronous programming paradigm.
However, this approach still requires a user-level I/O subsystem capable of
mitigating system-level blocking, which is crucial for the performance of
I/O-intensive applications. This technique offers a lightweight solution for
efficiently managing a larger number of concurrent sessions by minimizing
per-thread overhead.
In 2020, Karsten~\cite{karsten2020} demonstrated how this strategy supports a
synchronous programming style, effectively abstracting away the complexities
associated with managing continuations in asynchronous programming.

Among mainstream technologies, the Kotlin programming language introduces a new 
abstraction for managing coroutines and provides structured concurrency, which 
ensures that coroutines are \textit{scoped}, \textit{cancellable}, and 
\textit{coordinated} within a well-defined 
lifecycle~\cite{elizarov2021coroutines}.
Although this model embraces the \texttt{async}/\texttt{await} 
feature~\cite{async_await}, enabling non-blocking routines to mimic the 
structure of synchronous ones and allowing developers to reason about control 
flow sequentially, it still lacks a user-level I/O subsystem and relies on an 
explicit I/O dispatcher tied to a dedicated thread pool that frees worker 
threads for handling processing tasks.
Moreover, web templates using their own templating dialects, such as JSP, 
Thymeleaf, or Handlebars, are unable to take advantage of such constructs, as 
they typically support only a limited subset of the host library's API—most 
commonly just the \texttt{Iterable} interface.

Only Java virtual threads, introduced in
JEP~444~\footnote{\url{https://openjdk.org/jeps/444}}, adopt a similar approach
to Karsten's proposal, using user-mode threads to preserve a synchronous
programming model while still allowing blocking I/O operations without tying up
platform threads. When a virtual thread performs a blocking I/O operation, the
JVM intercepts the call and transparently parks the virtual thread, freeing the
underlying platform thread to perform other tasks. Once the I/O operation
completes, the virtual thread is rescheduled on a platform thread and resumes
execution. This mechanism is enabled by the JVM's integration with non-blocking
system calls at the OS level, allowing developers to write traditional
synchronous code while benefiting from the scalability of non-blocking I/O.


\subsubsection{Spring MVC}

\subsubsection{Spring WebFlux}

\subsubsection{Quarkus}

%%%%%%%%%%%%%%%%%%%%%%%%%%%%%%%%%%%%%%%%%%%%%%%%%%%%%%%%%%%%%%%%%%%%
%-------------------------------------------------------------------
%%%%%%%%%%%%%%%%%%%%%%%%%%%%%%%%%%%%%%%%%%%%%%%%%%%%%%%%%%%%%%%%%%%%

\subsection{Web Templates}

\textit{Web templates} have been the most widely adopted approach for
constructing dynamic HTML pages.
Web templates (such as JSP, Handlebars, or Thymeleaf), also known as
\textit{web views}~\cite{Fowler02,Alur01}, are based on
HTML documents augmented with template-specific markers (e.g., \texttt{<\%>},
\texttt{\{\{\}\}}, or \texttt{\$\{\}}), which represent \textit{dynamic}
information to be replaced at runtime with the results of corresponding
computations, producing the final HTML page.
The process of parsing and replacing these markers---i.e.,
\textit{resolution}---is the primary responsibility of the \textit{template
engine}~\cite{Parr04}.

One key characteristic of \textit{web templates} is their ability to receive a 
\textit{context object}---equivalent to the \textit{model} in the model-view 
design pattern~\cite{mvc88,Parr04}---which provides the data used to fill 
template placeholders at runtime.

Web templates can be distinguished by several properties, namely:
\begin{enumerate}
    \item Templating idiom
    \item Supported data model APIs
    \item Asynchronous support
    \item Type safety and HTML safety
    \item Progressive rendering
\end{enumerate}

Although some of the aforementioned characteristics apply to both server-side 
and client-side approaches, we focus solely on web template technologies for 
server-side rendering, as our work is centered on that approach.

Before detailing each of the aforementioned characteristics, Table~11 presents 
a breakdown of state-of-the-art template engines, classified according to the 
identified properties.

\subsubsection{Templating idiom}

Web templates are based on a \textit{domain-specific language}
(DSL)~\cite{landin1966next}, which defines a language tailored to a specific
\textit{domain}~\cite{evans2004domain}—in this case, HTML for expressing web
documents.
The DSL constrains the template's syntax and semantics to match the structure
and purpose of HTML.

DSLs can be divided in two types: \textit{external} or
\textit{internal}\cite{dslbook}.
\textit{External} DSLs are languages created without any affiliation to a
concrete programming language.
An example of an \textit{external} DSL is the regular expressions search
pattern\cite{thompson1968}, since it defines its own syntax without any
dependency of programming languages. 
On the other hand an \textit{internal} DSL is defined within a host programming
language as a library and tends to be limited to the syntax of the host language,
such as Java.
JQuery\cite{resig2007pro} is one of the most well-known examples of an internal
DSL in Javascript, designed to simplify HTML DOM\cite{dom} tree traversal and
manipulation.

Traditionally, web template technologies use an \textit{external} DSL to define
control flow constructs and data binding primitives. Early web template engines
such as JSP, ASP, Velocity, PHP, and others adopted this \textit{external} DSL
approach as their templating dialect.
For example, a \texttt{foreach} loop can be expressed in each technology using its 
own DSL:
\\\texttt{<\% for(String item : items) \%>} in JSP, \texttt{<\% For Each item 
In items \%>} in legacy ASP with VBScript, \texttt{\#foreach(\$item in \$items)} in 
Velocity, or \texttt{<?php foreach (\$items as \$item):?>} in PHP.

On the other hand, \textit{internal} DSLs for HTML allow templates to be defined 
directly within the \emph{host} language (such as Java, Kotlin, Groovy, Scala, or 
other general-purpose programming languages), rather than using text-based 
template files~\cite{carvalho2019hot}. In this case, a web template is not limited 
to templating constructs but may instead leverage any available primitive of the 
host language or any external API.

For example, in Java, you may iterate over data using a \texttt{for} statement,
or using the \texttt{java.util.streams} API, or any other JVM library such as
Guava, Varv, StreamEx, or others—according to your preferences.

Using an internal DSL can have several benefits over using textual templates:
\begin{enumerate}
  \item \emph{Type safety}: Because the templates are defined with the host
  programming language, the compiler can check the syntax and types of the
  templates at compile time, which can help catch errors earlier in the
  development process.

  \item \emph{IDE support}: Many modern IDEs provide code completion, syntax
  highlighting, and other features, which can make it easier to write and
  maintain templates.
  
  \item \emph{Flexibility}: Use all the features of the host programming language
  to generate HTML, can make it easier to write complex templates and reuse code.
  
  \item \emph{Integration}: Because the templates are defined in Java code, for
  example, you can easily integrate them with other Java code in your
  application, such as controllers, services, repositories and models.
  
\end{enumerate}

DSLs for HTML provide an API where methods or functions correspond to the names of
available HTML elements. 
These methods, also known as \textit{builders}, can be combined in a chain of
calls to mimic the construction of an HTML document in a fluent manner.
Martin Fowler\cite{dslbook} identifies three different patterns for combining
functions to create a DSL: 1) \textit{function sequence}; 2) \textit{nested
function}, and 3) \textit{method chaining}, which are illustrated in the snippets of
Figure~\ref{fig:dsl-idioms}.
\begin{figure}[h]
\centering

% Subfloat 1
\begin{minipage}[c]{0.33\linewidth}
  \centering
  \begin{lstlisting}[
    language=java,
    basicstyle=\scriptsize\ttfamily,
    numbers=none
  ]
html();
 head();
  title();text("JT");end();
 end();
 body();
  p();
   text("Hi JATL");
  end();
 end();
end();
  \end{lstlisting}
  \caption*{(a) Function sequence}
\end{minipage}
\hfil
% Subfloat 2
\begin{minipage}[c]{0.23\linewidth}
  \centering
  \begin{lstlisting}[
    language=java,
    basicstyle=\scriptsize\ttfamily,
    numbers=none
  ]
html(
 head(
  title("ST")
 ),
 body(
  p("Hi ScalaTags")
 )
);
  \end{lstlisting}
  \caption*{(b) Nested function}
\end{minipage}
\hfil
% Subfloat 3
\begin{minipage}[c]{0.32\linewidth}
  \centering
  \begin{lstlisting}[
    language=java,
    basicstyle=\scriptsize\ttfamily,
    numbers=none
  ]
html()
 .head()
  .title().text("HF").__()
 .__()
 .body()
  .p()
   .text("Hi HtmlFlow")
  .__()
 .__()
.__();
  \end{lstlisting}
  \caption*{(c) Method chaining}
\end{minipage}

\caption{Utilizing DSL for HTML libraries with JATL, ScalaTags, and HtmlFlow.}
\label{fig:dsl-idioms}
\end{figure}


From the three examples depicted in Figure~\ref{fig:dsl-idioms}, the \textit{nested
function} idiom used by \textbf{ScalaTags}, as shown in the snippet of
Figure~\ref{fig:dsl-idioms}.b), is the least verbose, requiring fewer statements than
the other two cases.
This approach of combining function calls is also utilized by \textbf{Hiccup}
Clojure Library and \textbf{j2html} Java library.
One reason for verbosity in \textit{function sequence} and \textit{method
chaining} approaches is the requirement of a dedicated HTML builder to emit the
closing tag, exemplified by \texttt{end()} in \textbf{JATL} (Figure~\ref{fig:dsl-idioms}.a) and
\texttt{\_\_()} in \textbf{HtmlFlow} (Figure~\ref{fig:dsl-idioms}.c).
But the \textit{nested function} approach comes with a notable drawback: \textbf{they
do not support PSSR} because the sequence of nested functions is evaluated
backward to the order in which they are written. 
In other words, arguments are evaluated before the functions are invoked. 
Taking the example in Figure~\ref{fig:dsl-idioms}.b), the \texttt{title()} function is
first evaluated, and its resulting paragraph becomes the argument for the
\texttt{head()} call, which, in turn, becomes the argument for \texttt{html()},
and so on.
If HTML is emitted as functions are called, it will print tags in reverse order. 
The aforementioned DSLs must collect resulting nodes into an internal data
structure, which is later traversed to produce the HTML output. Therefore, they
cannot progressively emit the output as builders are called.

Two other JVM libraries, \textbf{Groovy MarkupBuilder} and
\textbf{KotlinX.html}, also adopt a \textit{nested function} approach, but they
address the backward evaluation issue by implementing \textit{lazy evaluation}
of arguments~\cite{Landin65}, expressed in lambda expressions (also known as
function literals).
Due to the concise form of expressing lambdas with brackets (i.e.,
\texttt{\{\}}) in both Groovy and Kotlin, translating the template shown in
Figure~\ref{fig:dsl-idioms}.b) to use Groovy MarkupBuilder or KotlinX.html only
requires replacing parentheses with brackets for parent elements.
\textbf{HtmlFlow} also adopts this approach and provides a Kotlin-idiomatic API 
as an alternative to its original Java API.


%%%%%%%%%%%%%%%%%%%%%%%%%%%%%%%%%%%%%%%%%%%%%%%%%%%%%%%%%%%%%%%%%%%%
%-------------------------------------------------------------------
%%%%%%%%%%%%%%%%%%%%%%%%%%%%%%%%%%%%%%%%%%%%%%%%%%%%%%%%%%%%%%%%%%%%

\subsubsection{Supported data model APIs}

Restricted or unrestricted.

%%%%%%%%%%%%%%%%%%%%%%%%%%%%%%%%%%%%%%%%%%%%%%%%%%%%%%%%%%%%%%%%%%%%
%-------------------------------------------------------------------
%%%%%%%%%%%%%%%%%%%%%%%%%%%%%%%%%%%%%%%%%%%%%%%%%%%%%%%%%%%%%%%%%%%%

\subsubsection{Asynchronous support}
\label{sec:async-support}

Explicar Observable e modelos reactivos

Distinguish between template engines that support and do not support asynchronous data models.

1. Nem se quer Existe API para iterar sobre um Async data model. 
   E.g JStachio, .... apenas suportam interface Iterable. => Compilation Error.

2. Consegue usar qq API e.g Kotlinx.Html and Groovy  => usa mal => produzir HTML out of order

3. Sim suporta => well-formed HTML => Thymleaf e o HtmlFlow


One of the reasons for legacy web templates not supporting asynchronous APIs
is the absence of a unified standard calling convention for asynchronous calls.
While there is a single, straightforward way to use a synchronous API with a
direct style, where the result of a method call corresponds to its returned
value, there is no equivalent standard in the asynchronous approach. 
Instead, we may encounter various asynchronous conventions depending on the
programming language and runtime environment.
Some of these approaches include \emph{continuation-passing style}
(CPS)~\cite{scheme}, \textit{promises}~\cite{promise} async/await
idiom~\cite{async_await}, reactive streams \cite{ReactiveStreams}, Kotlin
Flow~\cite{kotlinlang}, and others.

Many \textit{general-purpose languages} (GPLs) have embraced the
\texttt{async}/\texttt{await} feature~\cite{async_await} enabling non-blocking
routines to mimic the structure of synchronous ones, allowing developers to
reason about instruction flow sequentially.
The simplicity and broad adoption of this programming model have led to its
incorporation into mainstream languages like C\#, JavaScript, Python, Perl,
Swift, Kotlin, and others, excluding Java.
However, implementing \texttt{async}/\texttt{a}wait requires compiler support to
translate \textit{suspension points} (i.e., \texttt{await} statements) into state
machines. Most template engines operate using an external DSL with their own
templating dialect (e.g., Thymeleaf, JSP, Jade, Handlebars, and others), which do not
inherently leverage asynchronous capabilities from their host GPLs.

%%%%%%%%%%%%%%%%%%%%%%%%%%%%%%%%%%%%%%%%%%%%%%%%%%%%%%%%%%%%%%%%%%%%
%-------------------------------------------------------------------
%%%%%%%%%%%%%%%%%%%%%%%%%%%%%%%%%%%%%%%%%%%%%%%%%%%%%%%%%%%%%%%%%%%%

\subsubsection{Type safety and HTML safety}

Another key characteristic of such DSLs is \textbf{HTML safety}, which refers to
whether they produce only valid HTML conforming to a well-formed document.
To ensure HTML safety, a DSL API should only allow combining calls to builders
that result in valid HTML. 
However, most HTML DSLs using the \textit{function sequence} or \textit{nested
function} approach cannot ensure HTML safety at compile time. 

The \textit{function sequence} approach, as illustrated in the snippet of
Figure~\ref{fig:dsl-idioms}.a), involves combining function calls as a sequence of
statements, making it challenging to restrict the order of statements. 
In the \textit{nested function} approach shown in
Figure~\ref{fig:dsl-idioms}.b), variable-length arguments are often used to
allow an undefined number of child elements, which cannot be strongly typed in
every programming language.

KotlinX.html, which utilizes the \textit{nested function} idiom with lazy
evaluation, mitigates this issue through function types with a receiver. 
In this approach, the receiver (i.e., \texttt{this} within the lambda) is
strongly typed and provides a set of methods corresponding to legal child
elements.
This enables KotlinX.html to enforce HTML safety by restricting the available
methods during compile time.

To achieve this KotlinX.html provides HTML builders using \textit{function
literals with receiver}~\cite{kotlinlang}.
In Kotlin, a block of code enclosed in curly braces \texttt{\{...\}} is known as
a \emph{lambda}, and can be used as an argument to
a function that expects a \emph{function literal}.
When we write, for example, \texttt{body \{ div \{ hr() \} \}}, we are
invoking the \texttt{body} function with a lambda as its argument. This
lambda, in turn, calls the \texttt{div} function with another lambda as an
argument that creates a horizontal row (i.e. \texttt{hr}).
Each call to an HTML builder (e.g., \texttt{body}, \texttt{div}, \texttt{hr})
creates the child element within the element generated by the outer function
call.

HtmlFlow provides two APIs: one in idiomatic Kotlin, similar to the \texttt{KotlinX.html} 
API, and another that employs the \textit{method chaining} idiom, as illustrated 
in the snippet of Figure~\ref{fig:dsl-idioms}.c).
In this approach, the receiver object is implicitly passed as an argument to
each method call, enabling subsequent methods to be invoked on the result of the
preceding one.
This facilitates the composition of methods, with each call building upon the
other.
Similar to KotlinX.html, HtmlFlow ensures HTML safety, restricting the available
HTML builders and attributes after the dot (\texttt{.}) operator.

%%%%%%%%%%%%%%%%%%%%%%%%%%%%%%%%%%%%%%%%%%%%%%%%%%%%%%%%%%%%%%%%%%%%
%-------------------------------------------------------------------
%%%%%%%%%%%%%%%%%%%%%%%%%%%%%%%%%%%%%%%%%%%%%%%%%%%%%%%%%%%%%%%%%%%%

\subsubsection{Progressive rendering}
