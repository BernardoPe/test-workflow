\subsection{Web Templates}

\textit{Web templates} have been the most widely adopted approach for
constructing dynamic HTML pages.
Web templates or \textit{web views}~\cite{Fowler02,Alur01} (e.g. JSP, Handlebars,
or Thymeleaf), are based on HTML documents augmented with template-specific
markers (e.g., \texttt{<\%>}, \texttt{\{\{\}\}}, or \texttt{\$\{\}}), which
represent \textit{dynamic} information to be replaced at runtime with the
results of corresponding computations, producing the final HTML page.
The process of parsing and replacing these markers---i.e.,
\textit{resolution}---is the primary responsibility of the \textit{template
  engine}~\cite{Parr04}.
One key characteristic of \textit{web templates} is their ability to receive a
\textit{context object}---equivalent to the \textit{model} in the model-view
design pattern~\cite{mvc88,Parr04}---which provides the data used to fill
template placeholders at runtime.
Web templates can be distinguished by several properties, namely:
\begin{enumerate}
  \item \textit{Domain-specific language} idiom
  \item Supported data model APIs
  \item Asynchronous support
  \item Type safety and HTML safety
  \item Progressive rendering
\end{enumerate}

Although some of the aforementioned characteristics apply to both
\textit{server-side} and \textit{client-side} approaches, we focus solely on
web template technologies for \textit{server-side rendering}, as our work is
centered on that approach. Before getting deep on each of the aforementioned
characteristics, Table~\ref{table:cmplibs} presents a breakdown of mainstream
template engines, classified according to the identified properties.

\begin{table}[h]
  \small
  \tabcolsep=0.1cm
  \def\arraystretch{1.2}
  \begin{tabular}{|c|c|c|c|c|c|c|}
    \hline
    \textbf{Library}
     & \textbf{DSL idiom}
     & \textbf{Data Model APIs}
     & \shortstack{\textbf{Asynchronous} \\\textbf{Support}}
     & \shortstack{\textbf{Type}         \\\textbf{Safety}}
     & \shortstack{\textbf{HTML}         \\\textbf{Safety}}
     & \shortstack{\textbf{Progressive}  \\\textbf{Rendering}}
    \\
    \hline
    \textbf{Freemarker}
     & External DSL
     & \texttt{Iterable}
     & \large{$\textcolor{red}{\times}$}
     & \large{$\textcolor{red}{\times}$}
     & \large{$\textcolor{red}{\times}$}
     & \large{$\textcolor{PineGreen}{\checkmark}$}
    \\
    \hline
    \textbf{JSP}
     & External DSL
     & \texttt{Iterable}
     & \large{$\textcolor{red}{\times}$}
     & \large{$\textcolor{red}{\times}$}
     & \large{$\textcolor{red}{\times}$}
     & \large{$\textcolor{red}{\times}$}
    \\
    \hline
    \textbf{JStachio}
     & External DSL
     & \texttt{Iterable}
     & \large{$\textcolor{red}{\times}$}
     & \textcolor{PineGreen}{\checkmark}$^{(4)}$
     & \large{$\textcolor{red}{\times}$}
     & \large{$\textcolor{PineGreen}{\checkmark}$}
    \\\hline
    \textbf{Pebble}
     & External DSL
     & \texttt{Iterable}
     & \large{$\textcolor{red}{\times}$}
     & \large{$\textcolor{red}{\times}$}
     & \large{$\textcolor{red}{\times}$}
     & \large{$\textcolor{PineGreen}{\checkmark}$}
    \\
    \hline
    \textbf{Qute}
     & External DSL
     & \texttt{Iterable}
     & \large{$\textcolor{red}{\times}$}
     & \large{$\textcolor{PineGreen}{*}$}$^{(5)}$
     & \large{$\textcolor{red}{\times}$}
     & \large{$\textcolor{PineGreen}{\checkmark}$}
    \\
    \hline
    \textbf{Rocker}
     & External DSL
     & \texttt{Iterable}
     & \large{$\textcolor{red}{\times}$}
     & \large{$\textcolor{red}{\times}$}
     & \large{$\textcolor{red}{\times}$}
     & \large{$\textcolor{PineGreen}{\checkmark}$}
    \\
    \hline
    \textbf{Thymeleaf}
     & External DSL
     & \shortstack{\texttt{Iterable} \\\texttt{Stream} \\\texttt{Publisher}$^{(1)}$}
     & \texttt{Publisher}$^{(1)}$
     & \large{$\textcolor{red}{\times}$}
     & \large{$\textcolor{red}{\times}$}
     & \large{$\textcolor{PineGreen}{\checkmark}$}
    \\
    \hline
    \textbf{Trimou}
     & External DSL
     & \texttt{Iterable}
     & \large{$\textcolor{red}{\times}$}
     & \large{$\textcolor{red}{\times}$}
     & \large{$\textcolor{red}{\times}$}
     & \large{$\textcolor{PineGreen}{\checkmark}$}
    \\
    \hline
    \textbf{Velocity}
     & External DSL
     & \shortstack{\texttt{Iterable} \\\texttt{Sequence}$^{(2)}$}
     & \large{$\textcolor{red}{\times}$}
     & \large{$\textcolor{red}{\times}$}
     & \large{$\textcolor{red}{\times}$}
     & \large{$\textcolor{PineGreen}{\checkmark}$}
    \\
    \hline
    \textbf{Clojure Hiccup}
     & Nested Eager
     & All
     & \large{$\textcolor{red}{\times}$}
     & \large{$\textcolor{PineGreen}{\checkmark}$}
     & \large{$\textcolor{red}{\times}$}
     & \large{$\textcolor{red}{\times}$}
    \\
    \hline
    \shortstack{\textbf{Groovy } \\\textbf{MarkupBuilder}}
     & Nested Lazy
     & All
     & \large{$\textcolor{red}{\times}$}
     & \large{$\textcolor{PineGreen}{\checkmark}$}
     & \large{$\textcolor{red}{\times}$}
     & \large{$\textcolor{PineGreen}{\checkmark}$}
    \\
    \hline
    \textbf{HtmlFlow}
     & Chaining
     & All
     & \large{$\textcolor{PineGreen}{\checkmark}$}
     & \large{$\textcolor{PineGreen}{\checkmark}$}
     & \large{$\textcolor{PineGreen}{\checkmark}$}
     & \large{$\textcolor{PineGreen}{\checkmark}$}
    \\
    \hline
    \textbf{j2html}
     & Nested Eager
     & All
     & \large{$\textcolor{red}{\times}$}
     & \large{$\textcolor{PineGreen}{\checkmark}$}
     & \large{$\textcolor{red}{\times}$}
     & \large{$\textcolor{red}{\times}$}
    \\
    \hline
    \textbf{KotlinX}
     & Nested Lazy
     & All
     & \large{$\textcolor{red}{\times}$}
     & \large{$\textcolor{PineGreen}{\checkmark}$}
     & \textcolor{PineGreen}{\checkmark}$^{(3)}$
     & \large{$\textcolor{PineGreen}{\checkmark}$}
    \\
    \hline
    \textbf{ScalaTags}
     & Nested Eager
     & All
     & \large{$\textcolor{red}{\times}$}
     & \large{$\textcolor{PineGreen}{\checkmark}$}
     & \large{$\textcolor{red}{\times}$}
     & \large{$\textcolor{red}{\times}$}
    \\
    \hline
  \end{tabular}
  \caption{
    Comparison of web template technologies in the Java ecosystem.
    \\$^{(1)}$ \texttt{Publisher} of Reactive Streams~\cite{ReactiveStreams}.
    Limited to a single model per web template.
    \\$^{(2)}$ \texttt{kotlin.sequences.Sequence}
    \\$^{(3)}$ Non-safety for HTML attributes.
    \\$^{(4)}$ Enforced via annotation processor that checks templates against typed interfaces at compile time.
    \\$^{(5)}$ Compile-time expression validation available via \texttt{@CheckedTemplate} and build-time metadata in Quarkus.
  }
  \label{table:cmplibs}
\end{table}

The first half of Table~\ref{table:cmplibs} lists template engines that use
their own templating dialects, referred to as \textit{external DSLs}. The
second half lists Java libraries that provide an \textit{internal DSL},
typically using a \textit{nested} or \textit{chaining} style to build HTML.
Note that, by leveraging the host language (e.g., Clojure, Groovy, Java,
Kotlin, or Scala) as the templating idiom, the latter impose no restrictions on
the data model and fully support all styles of data access APIs. They also
benefit from static type checking, which helps ensure \textit{type safety}.

%%%%%%%%%%%%%%%%%%%%%%%%%%%%%%%%%%%%%%%%%%%%%%%%%%%%%%%%%%%%%%%%%%%%
%-------------------------------------------------------------------
%%%%%%%%%%%%%%%%%%%%%%%%%%%%%%%%%%%%%%%%%%%%%%%%%%%%%%%%%%%%%%%%%%%%

\subsubsection{Domain-specific language idiom}

Web templates are based on a \textit{domain-specific language}
(DSL)~\cite{landin1966next}, which defines a language tailored to a specific
\textit{domain}~\cite{evans2004domain}—in this case, HTML for expressing web
documents. The DSL constrains the template's syntax and semantics to match the
structure and purpose of HTML.
DSLs can be divided in two types: \textit{external} or
\textit{internal}\cite{dslbook}. \textit{External} DSLs are languages created
without any affiliation to a concrete programming language. An example of an
\textit{external} DSL is the regular expressions search
pattern\cite{thompson1968}, since it defines its own syntax without any
dependency of programming languages. On the other hand an \textit{internal} DSL
is defined within a host programming language as a library and tends to be
limited to the syntax of the host language, such as Java.
JQuery\cite{resig2007pro} is one of the most well-known examples of an internal
DSL in Javascript, designed to simplify HTML DOM\cite{dom} tree traversal and
manipulation.

Traditionally, web template technologies use an \textit{external} DSL to define
control flow constructs and data binding primitives. Early web template engines
such as JSP, ASP, Velocity, PHP, and others adopted this \textit{external} DSL
approach as their templating dialect. For example, a \texttt{foreach} loop can
be expressed in each technology using its own DSL:
\begin{itemize}
  \item \texttt{<\% for(String item : items) \%>} in JSP
  \item \texttt{<\% For Each item In items \%>} in legacy ASP with VBScript
  \item \texttt{\#foreach(\$item in \$items)} in Velocity
  \item \texttt{<?php foreach (\$items as \$item):?>} in PHP
\end{itemize}

On the other hand, \textit{internal} DSLs for HTML allow templates to be
defined directly within the \emph{host} language (such as Java, Kotlin, Groovy,
Scala, or other general-purpose programming languages), rather than using
text-based template files~\cite{carvalho2020}. In this case, a web template
is not limited to templating constructs but may instead leverage any available
primitive of the host language or any external API.

Using an internal DSL can have several benefits over using textual templates:
\begin{enumerate}
  \item \emph{Type safety}: Because the templates are defined with the host
        programming language, the compiler can check the syntax and types of the
        templates at compile time, which can help catch errors earlier in the
        development process.

  \item \emph{IDE support}: Many modern IDEs provide code completion, syntax
        highlighting, and other features, which can make it easier to write and
        maintain templates.

  \item \emph{Flexibility}: Use all the features of the host programming language
        to generate HTML, can make it easier to write complex templates and reuse code.

  \item \emph{Integration}: Because the templates are defined in Java code, for
        example, you can easily integrate them with other Java code in your
        application, such as controllers, services, repositories and models.

\end{enumerate}

DSLs for HTML provide an API where methods or functions correspond to the names
of available HTML elements. These methods, also known as \textit{builders}, can
be combined in a chain of calls to mimic the construction of an HTML document
in a fluent manner. Martin Fowler\cite{dslbook} identifies three different
patterns for combining functions to create a DSL: 1) \textit{function
  sequence}; 2) \textit{nested function}, and 3) \textit{method chaining}, which
are illustrated in the snippets of Figure~\ref{fig:dsl-idioms}.
\begin{figure}[h]
  \centering

  % Subfloat 1
\begin{minipage}[c]{0.33\linewidth}
  \centering
  \begin{lstlisting}[
  language=java,
  basicstyle=\scriptsize\ttfamily,
  numbers=none
]
html();
 head();
  title();text("JT");end();
 end();
 body();
  p();
   text("Hi JATL");
  end();
 end();
end();
  \end{lstlisting}
    \caption*{(a) Function sequence}
  \end{minipage}
  \hfil
  % Subfloat 2
  \begin{minipage}[c]{0.23\linewidth}
    \centering
    \begin{lstlisting}[
    language=java,
    basicstyle=\scriptsize\ttfamily,
    numbers=none
  ]
html(
 head(
  title("ST")
 ),
 body(
  p("Hi ScalaTags")
 )
);
  \end{lstlisting}
    \caption*{(b) Nested function}
  \end{minipage}
  \hfil
  % Subfloat 3
  \begin{minipage}[c]{0.32\linewidth}
    \centering
    \begin{lstlisting}[
    language=java,
    basicstyle=\scriptsize\ttfamily,
    numbers=none
  ]
html()
 .head()
  .title().text("HF").__()
 .__()
 .body()
  .p()
   .text("Hi HtmlFlow")
  .__()
 .__()
.__();
  \end{lstlisting}
    \caption*{(c) Method chaining}
  \end{minipage}

  \caption{Utilizing DSL for HTML libraries with JATL, ScalaTags, and HtmlFlow.}
  \label{fig:dsl-idioms}
\end{figure}

From the three examples depicted in Figure~\ref{fig:dsl-idioms}, the
\textit{nested function} idiom used by \textbf{ScalaTags}, as shown in the
snippet of Figure~\ref{fig:dsl-idioms}.b), is the least verbose, requiring
fewer statements than the other two cases. This approach of combining function
calls is also utilized by \textbf{Hiccup} Clojure Library and \textbf{j2html}
Java library. One reason for verbosity in \textit{function sequence} and
\textit{method chaining} approaches is the requirement of a dedicated HTML
builder to emit the closing tag, exemplified by \texttt{end()} in \textbf{JATL}
(Figure~\ref{fig:dsl-idioms}.a) and \texttt{\_\_()} in \textbf{HtmlFlow}
(Figure~\ref{fig:dsl-idioms}.c). But the \textit{nested function} approach
comes with a notable drawback: \textbf{they do not support PSSR} because the
sequence of nested functions is evaluated backward to the order in which they
are written. In other words, arguments are evaluated before the functions are
invoked. Taking the example in Figure~\ref{fig:dsl-idioms}.b), the
\texttt{title()} function is first evaluated, and its resulting paragraph
becomes the argument for the \texttt{head()} call, which, in turn, becomes the
argument for \texttt{html()}, and so on. If HTML is emitted as functions are
called, it will print tags in reverse order. The aforementioned DSLs must
collect resulting nodes into an internal data structure, which is later
traversed to produce the HTML output. Therefore, they cannot progressively emit
the output as builders are called.

Two other JVM libraries, \textbf{Groovy MarkupBuilder} and
\textbf{KotlinX.html}, also adopt a \textit{nested function} approach, but they
address the backward evaluation issue by implementing \textit{lazy evaluation}
of arguments~\cite{Landin65}, expressed in lambda expressions (also known as
function literals). Due to the concise form of expressing lambdas with brackets
(i.e., \texttt{\{\}}) in both Groovy and Kotlin, translating the template shown
in Figure~\ref{fig:dsl-idioms}.b) to use Groovy MarkupBuilder or KotlinX.html
only requires replacing parentheses with brackets for parent elements.
\textbf{HtmlFlow} also adopts this approach and provides a Kotlin-idiomatic API
as an alternative to its original Java API.

%%%%%%%%%%%%%%%%%%%%%%%%%%%%%%%%%%%%%%%%%%%%%%%%%%%%%%%%%%%%%%%%%%%%
%-------------------------------------------------------------------
%%%%%%%%%%%%%%%%%%%%%%%%%%%%%%%%%%%%%%%%%%%%%%%%%%%%%%%%%%%%%%%%%%%%

\subsubsection{Supported data model APIs}

Template engines with dedicated templating dialects typically rely on specific
control flow constructs—such as a \texttt{foreach}-like statement—to iterate
over elements in a data source. In the Java ecosystem, \texttt{Iterable} is the
common supertype implemented by most data structures and serves as the standard
API for iteration. However, other standard interfaces, such as
\texttt{java.util.Stream}, also represent sequences of elements but are not
compatible with \texttt{Iterable}. This fragmentation becomes more evident when
considering other JVM languages, such as Kotlin, which introduces additional
abstractions like the \texttt{Sequence} interface—also incompatible with
\texttt{Iterable}. As a result, template engines based on \textit{external}
DSLs must explicitly support each of these interfaces to accommodate the
various iteration protocols across languages and libraries.

For example, Thymeleaf supports both Java \texttt{Iterable} and \texttt{Stream}
iteration protocols but is incompatible with Kotlin's \texttt{Sequence}
interface.
Velocity, on the other hand, uses internal reflection-based template processing
that allows it to iterate over any type defining an \texttt{iterator()} method
returning an \texttt{Iterator}. This enables support for both Java
\texttt{Iterable} and Kotlin \texttt{Sequence}, but it cannot handle Java
\texttt{Stream} since \texttt{Stream} does not expose a compatible
\texttt{iterator()} method.
The remaining analyzed template engines that use external DSLs support only Java
\texttt{Iterable} as their sole iteration protocol.

On the other hand, template engines that employ an \textit{internal} DSL do not
face such limitations, as they can leverage any available API within the host
programming language. This flexibility extends beyond the standard library to
include third-party APIs as well. For example, in Java, developers can iterate
over data using a traditional \texttt{for} loop, the \texttt{java.util.stream}
API, or external libraries such as Guava, Vavr, StreamEx, and others—depending
on their specific needs or preferences.

%%%%%%%%%%%%%%%%%%%%%%%%%%%%%%%%%%%%%%%%%%%%%%%%%%%%%%%%%%%%%%%%%%%%
%-------------------------------------------------------------------
%%%%%%%%%%%%%%%%%%%%%%%%%%%%%%%%%%%%%%%%%%%%%%%%%%%%%%%%%%%%%%%%%%%%

\subsubsection{Asynchronous support}
\label{sec:async-support}

One of the reasons for legacy web templates not supporting asynchronous APIs is
the absence of a unified standard calling convention for asynchronous calls.
While there is a single, straightforward way to use a synchronous API with a
direct style, where the result of a method call corresponds to its returned
value, there is no equivalent standard in the asynchronous approach. Instead,
we may encounter various asynchronous conventions depending on the programming
language and runtime environment. Some of these approaches include
\emph{continuation-passing style} (CPS)~\cite{scheme},
\textit{promises}~\cite{promise}, async/await idiom~\cite{async_await},
or \textit{suspend functions}~\cite{elizarov2021coroutines}.

Many \textit{general-purpose languages} (GPLs) have embraced the
\texttt{async}/\texttt{await} feature~\cite{async_await} enabling non-blocking
routines to mimic the structure of synchronous ones, allowing developers to
reason about instruction flow sequentially. The simplicity and broad adoption
of this programming model have led to its incorporation into mainstream
languages like C\#, JavaScript, Python, Perl, Swift, Kotlin, and others,
excluding Java. However, implementing \texttt{async}/\texttt{a}wait requires
compiler support to translate \textit{suspension points} (i.e., \texttt{await}
statements) into state machines. Most template engines operate using an
external DSL with their own templating dialect (e.g., Thymeleaf, JSP, Jade,
Handlebars, and others), which do not inherently leverage asynchronous
capabilities from their host programming languages.

An alternative approach to asynchronous programming involves handling data as
a sequence of asynchronous events, commonly known as a \textit{reactive stream}.
Instead of materializing all items in memory, reactive streams emit values over
time, allowing clients to subscribe and consume data incrementally as it becomes
available.
This idea was first proposed by Meijer~\cite{rx-observable} in the .NET
framework, leading to the development of \textit{Reactive Extensions} (Rx).
Subsequently, several alternative implementations emerged in the Java ecosystem,
such as Rx Java~\cite{rxjava}, Project Reactor~\cite{projectreactor}, or the
Akka Streams~\cite{akka} library.
Later, due to the proliferation of alternative implementations, the need arose to
create a common interface to ensure interoperability between different reactive
stream libraries. This led to the formation of a working group that included
engineers from notable companies such as Lightbend, Pivotal, and Netflix.
The group began defining the \textit{Reactive Streams}
specification~\cite{ReactiveStreams}, which established a standard for
asynchronous stream processing using non-blocking, backpressure-aware data
pipelines.
The more recent Mutiny~\cite{mutiny2021}, while conforming to the
\textit{Reactive Streams} standard, proposes a different approach that avoids
complex optimizations, resulting in more straightforward and easier-to-maintain
code.

Thymeleaf was one of the first web template engines—and remains the only one
among external DSLs for HTML—to support reactive streams based on the
\texttt{Publisher} model. However, it is limited to using a single
\texttt{Publisher} within a given template.

%%%%%%%%%%%%%%%%%%%%%%%%%%%%%%%%%%%%%%%%%%%%%%%%%%%%%%%%%%%%%%%%%%%%
%-------------------------------------------------------------------
%%%%%%%%%%%%%%%%%%%%%%%%%%%%%%%%%%%%%%%%%%%%%%%%%%%%%%%%%%%%%%%%%%%%

\subsubsection{HTML safety}

Another key characteristic of such DSLs is \textit{HTML safety}, which refers
to whether they produce only valid HTML conforming to a well-formed document.
To ensure HTML safety, a DSL API should only allow combining calls to builders
that result in valid HTML. However, most HTML DSLs using the \textit{function
  sequence} or \textit{nested function} approach cannot ensure HTML safety at
compile time.

The \textit{function sequence} approach, as illustrated in the snippet of
Figure~\ref{fig:dsl-idioms}.a), involves combining function calls as a sequence
of statements, making it challenging to restrict the order of statements. In
the \textit{nested function} approach shown in Figure~\ref{fig:dsl-idioms}.b),
variable-length arguments are often used to allow an undefined number of child
elements, which cannot be strongly typed in every programming language.

KotlinX.html, which utilizes the \textit{nested function} idiom with lazy
evaluation, mitigates this issue through function types with a receiver. In
this approach, the receiver (i.e., \texttt{this} within the lambda) is strongly
typed and provides a set of methods corresponding to legal child elements. This
enables KotlinX.html to enforce HTML safety by restricting the available
methods during compile time.

To achieve this KotlinX.html provides HTML builders using \textit{function
  literals with receiver}~\cite{kotlinlang}. In Kotlin, a block of code enclosed
in curly braces \texttt{\{...\}} is known as a \emph{lambda}, and can be used
as an argument to a function that expects a \emph{function literal}. When we
write, for example, \texttt{body \{ div \{ hr() \} \}}, we are invoking the
\texttt{body} function with a lambda as its argument. This lambda, in turn,
calls the \texttt{div} function with another lambda as an argument that creates
a horizontal row (i.e. \texttt{hr}). Each call to an HTML builder (e.g.,
\texttt{body}, \texttt{div}, \texttt{hr}) creates the child element within the
element generated by the outer function call.

HtmlFlow provides two APIs: one in idiomatic Kotlin, similar to the
\texttt{KotlinX.html} API, and another that employs the \textit{method
  chaining} idiom, as illustrated in the snippet of
Figure~\ref{fig:dsl-idioms}.c). In this approach, the receiver object is
implicitly passed as an argument to each method call, enabling subsequent
methods to be invoked on the result of the preceding one. This facilitates the
composition of methods, with each call building upon the other. Similar to
KotlinX.html, HtmlFlow ensures HTML safety, restricting the available HTML
builders and attributes after the dot (\texttt{.}) operator.

%%%%%%%%%%%%%%%%%%%%%%%%%%%%%%%%%%%%%%%%%%%%%%%%%%%%%%%%%%%%%%%%%%%%
%-------------------------------------------------------------------
%%%%%%%%%%%%%%%%%%%%%%%%%%%%%%%%%%%%%%%%%%%%%%%%%%%%%%%%%%%%%%%%%%%%

\subsubsection{Progressive rendering}

Pursuing the optimal user experience for web users has been a consistent goal
since the inception of the World Wide Web. Consequently, various approaches
have been explored to deliver the initial meaningful content to the end-user as
promptly as possible, while the remaining content \textit{progressively} (or
incrementally) loads as the server streams the HTML content.

\textit{\textbf{Progressive rendering}} and \textit{\textbf{progressive loading}}
encompass different concepts. The former pertains to the \textit{dynamic} content of a
\textit{dynamic web page}, encompassing elements with logic and placeholders
that are fulfilled by data from an object \textit{model} constructed at runtime.
On the other hand, the latter is associated with \textit{render-blocking
  resources} such as scripts, stylesheets, and HTML imports, which may hinder the
browser from rendering page content to the screen~\cite{progressive-2022}.
The notion of \textit{progressive rendering} has typically been aligned with
\textit{client-side rendering} (CSR), where a single HTML page with static
content is delivered upfront, while the dynamic content is fetched as the data
becomes available to complete the web page.
However, despite being overlooked, HTTP and browsers were designed from their
inception to also support this feature in the context of SSR approaches,
which offers the advantage of not being dependent on
\textit{rendering-blocking resources}, such as the required JavaScript for CSR.
% \textit{Progressive server-side rendering} (PSSR) is the
% ability of streaming HTML content to the client in chunks or data frames, as it
% gets resolved on the server at runtime.
% This lets the user-interface to be rendered incrementally (i.e.
% \textit{progressively}) by the browser in line with the availability of the data.
% However, most SSR template engines do not inherently support this feature at the
% template level, requiring developers to manually break down web pages into
% fragments or partials to enable PSSR.
An example of this limitation in SSR web templates was highlighted by Jeff
Atwood in 2005, who criticized Microsoft ASP.NET for loading the entire web
page into memory before sending any data to the browser~\cite{pssr2005}.
% This practice delays the display of content to the user until the server
% completes its processing, contradicting the original design intent of HTML to
% render progressively as content is received.
% Notably, both Netscape and Internet Explorer were capable of rendering partial
% HTML content from their inception.
Despite historical critiques and HTML's inherent capabilities, most Web
application frameworks, such as ASP.Net, Express.js, Spring, and others,
persistently lack support for progressive rendering, leading to the appearance
of alternative techniques leveraging client-side JavaScript. Since 2007,
various patents have addressed the PSSR issue, with Microsoft's patent enabling
the infinite scrolling technique by displaying a single page of
results~\cite{scroll2007}, and Yahoo's patent focusing on differentiating
elements based on their position relative to the visible
portion~\cite{schiller2007progressive}.

Former techniques all describe methods of progressively adding content to a web
page depending on client-side JavaScript.
Java Thymeleaf, the default SSR template engine for Spring web servers,
added support for PSSR in 2018, through the use of a specific non-blocking
Spring \texttt{ViewResolver} driver and without requiring client-side
JavaScript.
Carvalho~\cite{carvalho2023async} introduced an SSR solution that manages
multiple data models and asynchronous APIs while ensuring well-formed HTML,
PSSR, and non-blocking template resolution. That solution builds upon HtmlFlow,
a Java DSL for HTML, with an internal processing mechanism that streams HTML in
chunks as data items become available from data sources, without requiring
client-side JavaScript. Like Thymeleaf, their proposal avoids the use of
client-side JavaScript; however, unlike Thymeleaf, it supports a wide range of
asynchronous APIs and multiple data sources, not limited to the
\texttt{Publisher} API. However, the main counter-argument is the non-trivial
management of the \emph{resume} callback in
continuations~\cite{von2003events,callbackhell}, which is used to linearize the
execution flow between asynchronous calls.