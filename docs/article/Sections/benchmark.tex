\section{Benchmark Implementation}

The benchmark is designed with a modular architecture, separating the
\textit{view} and \textit{model} layers from the \textit{controller} layer
\cite{Bucanek2009}, which allows for easy extension and integration of new
template engines and frameworks. It also includes a set of tests to ensure the
correctness of implementations and to validate the \textit{HTML} output.

The benchmark includes two different data models, defined as \texttt{Presentation} and 
\texttt{Stock} shown in \autoref{lst:data-models}. 

\lstdefinestyle{listingstyle}{
  backgroundcolor=\color{lightgray!20},
  basicstyle=\ttfamily\small,
  breaklines=true,
  frame=single,
  captionpos=b,
  numbers=left,
  numberstyle=\tiny,
  keywordstyle=\color{blue},
  commentstyle=\color{gray},
  stringstyle=\color{orange},
}

\lstset{style=listingstyle}

\begin{lstlisting}[language=Kotlin, caption={Data Models}, label={lst:data-models}]
data class Presentation(
    val id: Long,
    val title: String, 
    val speakerName: String,
    val summary: String
)

data class Stock(
    val name: String,
    val name2: String,
    val url: String,
    val symbol: String,
    val price: Double, 
    val change: Double, 
    val ratio: Double
)
\end{lstlisting}

The application's repository contains a list of 10 instances of the
\texttt{Presentation} class and 20 instances of the \texttt{Stock} class. Each list is used
to generate a respective HTML view. Although the instances are kept in memory,
the repository uses the \texttt{Observable} class from the \textit{RxJava}
library to interleave list items with a delay of 1 millisecond. This delay
promotes context switching and frees up the calling thread to handle other
requests in non-blocking scenarios, mimicking actual I/O operations.

By using the \texttt{blockingIterable} method of the \texttt{Observable}
class, we provide a blocking interface for template engines that do not support
asynchronous data models, while still simulating the asynchronous nature of the
data source to enable PSSR\@. Template engines that do not support non-blocking
I/O for PSSR include KotlinX, Rocker, JStachio, Pebble, Freemarker, Trimou, and
Velocity. HtmlFlow supports non-blocking I/O through suspendable templates and
asynchronous rendering, while Thymeleaf enables it using the
\texttt{ReactiveDataDriverContextVariable} in conjunction with a non-blocking
Spring \texttt{ViewResolver}.

The aforementioned blocking template engines are used in
the context of Virtual Threads or alternative coroutine dispatchers, allowing
the handler thread to be released and reused for other requests.

The Spring WebFlux core implementation uses Project Reactor to support a reactive
programming model: each method returns a \texttt{Flux<String>} as the response
body, which acts as a publisher that progressively streams the HTML content to
the client. It also includes methods using Kotlin coroutines and other
asynchronous mechanisms supported by template engines, such as the
\texttt{writeAsync} method from HtmlFlow, which enables non-blocking I/O
using \textit{continuations}. 
The Spring MVC implementation uses handlers based solely on the blocking
interface of the \texttt{Observable} class. To enable PSSR in this context, we utilize
the \texttt{StreamingResponseBody} interface, which allows the application to
write directly to the response \texttt{OutputStream} without blocking the servlet
container thread. According to the Spring documentation, this class is \textit{a
controller method return value type for asynchronous request processing where
the application can write directly to the response OutputStream without holding
up the Servlet container thread} \cite{Spring-StreamingResponseBody}.

In Spring MVC, \texttt{StreamingResponseBody} enables asynchronous writing relative to
the request-handling thread, but the underlying I/O remains
blocking—specifically the writes to the \texttt{OutputStream}. When using Virtual
Threads, the I/O operations are more efficient when compared to platform threads,
as they are executed in the context of a lightweight thread. Most of the computation is done in a separate
thread from the one that receives each request; we use a thread pool
\texttt{TaskExecutor} to process requests, allowing the application to scale and
handle multiple clients more efficiently as opposed to the default
\texttt{TaskExecutor} implementation, which tries to create a thread for each request.

However, the Spring MVC implementation does not effectively support PSSR for
these templates, as HTML content is not streamed progressively to the client.
This is because the response is only sent once the content written to the
\texttt{OutputStream} exceeds the output buffer size, which defaults to 8KB\@. As a
result, the client receives the response only after the entire HTML content is
rendered, defeating the purpose of PSSR in this context.

The Quarkus implementation also uses handlers based on the blocking interface
of the \texttt{Observable} class. It implements the \texttt{StreamingOutput} interface
from the JAX-RS specification to enable PSSR, allowing HTML content to be
streamed to the client. While \texttt{StreamingOutput} also uses blocking I/O, it
operates on Vert.x worker threads, which prevents blocking of the event loop.
When Virtual Threads are used, the I/O operations are handled efficiently, as
they are executed in lightweight threads.

The Quarkus implementation supports PSSR for these templates by configuring the
response buffer size in the \textit{application.properties} file. The default
buffer size is 8KB, but we reduced it to 512 bytes, which allows the response
to be sent to the client progressively as the HTML content is rendered.
