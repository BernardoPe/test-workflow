\section{Introduction}

% Falta adicionar referências

Modern web applications rely on different rendering strategies to optimize
performance, user experience, and \textit{scalability}. The two most dominant
approaches are \textit{Server-Side Rendering (SSR)} and \textit{Client-Side
    Rendering (CSR)}. SSR generates HTML content on the server before sending it to
the client, resulting in a faster \textit{First Contentful Paint (FCP)}
~\cite{Edgar2024-FCP} and better Search Engine Optimization (SEO). However, SSR
can increase server load and reduce \textit{throughput}\footnote{The number of
    requests the server can handle per second (RPS)} since each request requires
additional processing before responding. In contrast, CSR shifts the rendering
workload to the browser—the server initially sends a minimal HTML document with
JavaScript, which dynamically loads the page content. While CSR reduces the
server’s burden, it can lead to a slower \textit{FCP}, as users must wait for
JavaScript execution before meaningful content appears.

\textit{Progressive Server-Side Rendering (PSSR)} combines benefits from both SSR and CSR
by streaming HTML content progressively. This technique enhances user-perceived performance by
allowing progressive rendering as data becomes available. In this respect,
PSSR is similar to CSR in that the server initially sends a minimal HTML
document to the client and subsequently streams additional HTML fragments.
However, unlike CSR, PSSR retains all rendering responsibilities on the server side,
thereby reducing the load on the client. As a result, the client does
not need to execute JavaScript or make additional requests to retrieve page content.

Efficient thread utilization is crucial for low-thread servers. These servers
must handle many concurrent requests without being overwhelmed, which
necessitates asynchronous, non-blocking request handling
~\cite{carvalho2023async}. By leveraging asynchronous I/O operations—such as
database queries and API calls, servers can avoid blocking threads while
waiting for data, thereby maximizing throughput and responsiveness. To support
this non-blocking architecture, PSSR implementations require template engines
that are compatible with asynchronous data models and I/O. Some modern template
engines, such as HtmlFlow~\cite{htmlflow} and Thymeleaf~\cite{thymeleaf}, have
been designed with these capabilities in mind. However, many traditional
template engines—particularly those using external domain-specific languages
~\cite{Fowler03} (DSLs)—still depend on blocking interfaces like
\texttt{Iterable} for data processing. This blocking behavior forces server
threads to remain idle until the entire HTML output is ready, undermining the
performance benefits of non-blocking I/O and limiting scalability in
high-concurrency environments. 

With the introduction of \textit{Virtual Threads}~\footnote{\url{https://openjdk.org/jeps/444}}
in Java 21, it is now possible to execute blocking I/O operations in a
scalable, lightweight manner. This capability allows traditional template
engines—often reliant on blocking interfaces—to operate efficiently in
high-concurrency, non-blocking environments without requiring complex
asynchronous programming models. As a result, Progressive Server-Side Rendering
(PSSR) can now be implemented using familiar HTML templates, 
simplifying development and improving maintainability.

This work explores the current landscape of non-blocking PSSR, focusing on two
primary paradigms: \textit{reactive programming} and \textit{coroutines}, both
of which have traditionally been used to achieve asynchronous I/O. As an
alternative, we investigate whether Java’s Virtual Threads can offer comparable
performance while preserving the simplicity of synchronous code. Section 2
reviews the state-of-the-art in PSSR and template engine design. Section 3
outlines the limitations of conventional engines in asynchronous settings and
presents our proposed approach. Section 4 details the benchmark methodology,
followed by the results and analysis in Section 5. Conclusions are presented in Section 6.